\iffalse
\documentclass{standalone}
\usepackage{tikz}
\usetikzlibrary{positioning,shapes,arrows}

\begin{document}
\fi

\tikzstyle{block} = [draw, rectangle, minimum height=3em, minimum width=6em]
\tikzstyle{sum} = [draw, circle, node distance=1cm]
\tikzstyle{input} = [coordinate]
\tikzstyle{output} = [coordinate]
\tikzstyle{pinstyle} = [pin edge={to-,thin,black}]
\tikzstyle{every node}=[font=\scriptsize]

% The block diagram code is probably more verbose than necessary
\begin{tikzpicture}[arrows=-triangle 45, node distance=2cm]
    % We start by placing the blocks
    \node [input, name=input] {};
    \node [sum, right of=input] (sum) {};
    \node [block, right of=sum] (controller) {Controller};
    \node [block, right of=controller, node distance=3cm] (system) {System};
    % We draw an edge between the controller and system block to 
    % calculate the coordinate u. We need it to place the measurement block. 
    \draw [] (controller) -- node[name=u] {} (system);
    \node [output, right of=system] (output) {};
    \coordinate [below of=u] (tmp);

    % Once the nodes are placed, connecting them is easy. 
    \draw [draw] (input) -- node {} (sum);
    \draw [] (sum) -- node {} (controller);
    \draw [] (system) -- node [name=y] {}(output);
    \draw [] (y) |- (tmp) -| node[pos=0.99] {} 
        node [near end] {} (sum);
\end{tikzpicture}





\iffalse
\end{document}
\fi