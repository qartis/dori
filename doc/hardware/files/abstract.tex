%% Copyright 1998 Pepe Kubon
%%
%% `abstract.tex' --- abstract for thes-full.tex, thes-short-tex from
%%                    the `csthesis' bundle
%%
%% You are allowed to distribute this file together with all files
%% mentioned in READ.ME.
%%
%% You are not allowed to modify its contents.
%%

%%%%%%%%%%%%%%%%%%%%%%%%%%%%%%%%%%%%%%%%%%%%%%%%%
%
%       Abstract 
%
%%%%%%%%%%%%%%%%%%%%%%%%%%%%%%%%%%%%%%%%%%%%%%%%

\prefacesection{Abstract}
\textsc{DORI} (Distributed Outdoor Robotic Instruments) is a remotely controlled vehicle that is designed to simulate a planetary exploration mission. \textsc{DORI} is equipped with over 20 environmental sensors and can perform basic data analysis, logging and remote upload. The individual components are distributed across a fault-tolerant bus for redundancy. A partial sensor list includes atmospheric pressure, rainfall, wind speed, GPS, gyroscopic inertia, linear acceleration, magnetic field strength, temperature, laser and ultrasonic distance sensing, as well as digital audio and video capture. The project uses recycled consumer electronics devices as a low-cost source for sensor components. This report describes the hardware design of DORI including sensor electronics, embedded firmware, and physical construction.











